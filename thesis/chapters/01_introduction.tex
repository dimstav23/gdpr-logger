% !TeX root = ../main.tex
\chapter{Introduction}\label{chapter:introduction}

The General Data Protection Regulation (GDPR), enacted in 2018, has fundamentally reshaped how organizations handle personal data, imposing strict requirements on data privacy, security, and user control. Designed to enhance individuals' rights over their data, GDPR mandates principles such as the right to be forgotten, data portability, and explicit consent, introducing significant challenges for database management. Traditional database engines, particularly key-value stores, must be re-evaluated to accommodate GDPR’s demands, as they were not originally designed for granular data erasure, extensive metadata tracking, or compliance auditing. This has led to the need for architectural redesigns, resulting in performance overheads, storage inefficiencies, and increased complexity in data retrieval and deletion processes. Consequently, organizations must reconsider how they implement and optimize database systems to ensure regulatory compliance without compromising efficiency.

\begin{enumerate}
        \item \textbf{Context of the Project}:
              \begin{itemize}
                      \item \textbf{Overview of GDPR}:
                            Introduce the General Data Protection Regulation (GDPR), its goals, and the impact it has had on database systems.
                      \item \textbf{Impact on Database Engines}:
                            Explain why traditional database engines, especially key-value stores, are being rethought in light of GDPR (e.g., need for redesign, performance overheads, metadata explosion).
              \end{itemize}
        \item \textbf{Motivation for the Project}:
              \begin{itemize}
                      \item \textbf{Challenges Introduced by GDPR}:
                            Highlight the practical challenges such as significant performance overheads, the need for tamper-proof audit trails, and the discontinuation of some applications due to compliance costs.
                      \item \textbf{Gap in Current Solutions}:
                            Discuss the absence of a uniform, secure, and efficient approach to enforce GDPR policies on key-value stores.
                      \item \textbf{Why a Logging System?}:
                            Emphasize the role of audit logging in ensuring compliance and the specific need for a tamper-evident, encrypted, and efficient logging mechanism.
              \end{itemize}
        \item \textbf{State-of-the-Art}:
              \begin{itemize}
                      \item \textbf{Existing Research and Solutions}:
                            Summarize the most important research work related to GDPR compliance, tamper-proof logging systems, and modifications to database engines.
                      \item \textbf{Critical Analysis}:
                            Identify strengths and weaknesses in current approaches, and discuss why none fully address the specific challenges posed by GDPR.
              \end{itemize}
        \item \textbf{Establishing the Research Gap}:
              \begin{itemize}
                      \item \textbf{Shortcomings of Current Approaches}:
                            Clearly articulate what is missing in the existing literature or solutions (e.g., balancing high performance with security and tamper evidence in logging).
                      \item \textbf{Need for a New Approach}:
                            Explain why a specialized component within a GDPR enforcement engine is necessary.
              \end{itemize}
        \item \textbf{Problem Statement}:
              \begin{itemize}
                      \item \textbf{Core Challenge}:
                            Formally define the problem the thesis aims to solve - designing and implementing a high-performance, secure, and tamper-proof logging system for key-value stores in a GDPR-compliant context.
                      \item \textbf{Scope of the Work}:
                            Outline the boundaries of the work and how it fits within the larger GDPR enforcement engine.
              \end{itemize}
        \item \textbf{High-Level Approach and Design}:
              \begin{itemize}
                      \item \textbf{Overview of the Proposed System}:
                            Introduce the logging system as an append-only mechanism designed to log all transactions on personal data.
                      \item \textbf{Key Architectural Components}:
                            Briefly describe components like encryption, compression, tamper-evident structures, and structured logging.
                      \item \textbf{Position in the Overall GDPR Enforcement Engine}:
                            Explain how your component integrates as part of the proxy system in front of key-value stores.
              \end{itemize}
        \item \textbf{Implementation Overview}:
              \begin{itemize}
                      \item \textbf{Technological Choices}:
                            Provide a high-level description of the technologies and programming paradigms you plan to use.
                      \item \textbf{Development Process}:
                            Summarize the stages of development, including prototyping, coding, and testing.
              \end{itemize}
        \item \textbf{Evaluation Overview}:
              \begin{itemize}
                      \item \textbf{Performance Metrics}:
                            Describe how you plan to measure high throughput and low interference with ongoing logging.
                      \item \textbf{Security and Tamper-Evidence}:
                            Outline the methods for validating the tamper-proof and encryption features.
                      \item \textbf{Compliance Reporting}:
                            Briefly mention how the system supports compliance reporting and bulk data exports.
              \end{itemize}
        \item \textbf{Impact Summary}:
              \begin{itemize}
                      \item \textbf{Practical Implications}:
                            Discuss how the proposed logging system could influence the design of GDPR-compliant database systems.
                      \item \textbf{Broader Benefits}:
                            Consider benefits such as improved security, easier compliance verification, and potential industry adoption.
              \end{itemize}
        \item \textbf{Key Contributions}:
              \begin{itemize}
                      \item \textbf{Bullet-Point List of Contributions}:
                            \begin{itemize}
                                    \item A novel tamper-proof logging mechanism tailored for key-value stores under GDPR constraints.
                                    \item A design that balances high performance with robust security and auditability.
                                    \item Integration strategies within an overarching GDPR enforcement engine.
                                    \item An evaluation framework that demonstrates the system’s efficacy in both performance and compliance contexts.
                            \end{itemize}
              \end{itemize}
\end{enumerate}