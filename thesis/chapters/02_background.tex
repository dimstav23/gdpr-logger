% !TeX root = ../main.tex
\chapter{Background}\label{chapter:background}

The following could be the structure of my Background chapter

\begin{enumerate}
        \item \textbf{Overview of GDPR and Its Implications}:
              \begin{itemize}
                      \item \textbf{GDPR Fundamentals}:
                            Present the main principles and requirements of the GDPR, emphasizing aspects relevant to data logging and auditing.
                      \item \textbf{Impact on Data Systems}:
                            Explain how GDPR has forced a redesign of traditional database and storage engines.
              \end{itemize}
        \item \textbf{Key-Value Stores and Their Role in Modern Data Systems}:
              \begin{itemize}
                      \item \textbf{Introduction to Key-Value Stores}:
                            Provide an overview of key-value databases, their typical use cases, and their advantages.
                      \item \textbf{Challenges Under GDPR}:
                            Discuss the specific challenges key-value stores face under GDPR (e.g., metadata explosion, compliance overheads).
              \end{itemize}
        \item \textbf{Tamper-Proof Logging Systems}:
              \begin{itemize}
                      \item \textbf{Definition and Importance}:
                            Define what makes a logging system “tamper-proof” and why this is crucial for compliance.
                      \item \textbf{Techniques and Approaches}:
                            Review existing techniques (e.g., append-only logs, cryptographic hashes, Merkle trees) used to secure logs.
                      \item \textbf{Relevance to GDPR Compliance}:
                            Explain how these techniques help in ensuring auditability and integrity of personal data logs.
              \end{itemize}
        \item \textbf{Security and Cryptography Basics}:
              \begin{itemize}
                      \item \textbf{Encryption Methods}:
                            Provide necessary background on encryption techniques that might be employed in the logging system.
                      \item \textbf{Digital Signatures and Hashing}:
                            Introduce the basic concepts of digital signatures and cryptographic hashing as they relate to tamper evidence.
                      \item \textbf{Compression Techniques}:
                            Briefly cover methods for compressing log data to handle high throughput without sacrificing efficiency.
              \end{itemize}
        \item \textbf{Existing Systems and Comparative Analysis}:
              \begin{itemize}
                      \item \textbf{Review of Related Systems}:
                            Present an overview of current solutions and research projects that deal with GDPR compliance or secure logging.
                      \item \textbf{Lessons Learned}:
                            Highlight key takeaways from these systems that have informed the design decisions.
              \end{itemize}
        \item \textbf{Technical Prerequisites for Understanding the Thesis}:
              \begin{itemize}
                      \item \textbf{Terminology and Concepts}:
                            Define key terms and concepts that will be frequently used in later chapters.
                      \item \textbf{Underlying Technologies}:
                            Offer a primer on any specific technologies (e.g., specific encryption libraries, database technologies) essential for understanding the implementation.
              \end{itemize}
\end{enumerate}